\section{Preliminaries}\label{sec:preliminaries}
\subsection{Temporal Logics}
Temporal Logics (TL) are a straightforward way to specify and verify properties of infinite behaviour of reactive systems, e.g. a transition system. TL extend either propositional or predicate logic with modalities, where the modalities are said to be \emph{time-abstract}, as they allow one to specify the order in which state labels occur during an execution of a system~\cite{baier2008principles}. Furthermore one can specify if some state labels will \emph{eventually} or \emph{always} occur during an execution. The following two elementary temporal modalities are included in most TL~\cite{baier2008principles}:
\begin{align*}
    &\F \quad \text{``eventually'' (some time in the future)}\\
    &\G \quad \text{``always'' (now and forever in the future)}
\end{align*}
\emph{Eventually} and \emph{always} will be explained more in-depth during this section. 

The time in TL can be either \emph{linear} or \emph{branching}, where in the linear view at each moment there is a single successor and in branching view time may split into multiple branches. A TL that is based on the branching view is the Computation Tree Logic (CTL) and one that is based on the linear view is Linear Temporal Logic (LTL)~\cite{baier2008principles}. In the remaining of this paper the focus will be on LTL that extends propositional logic, as the information flow policy language introduced in \autoref{sec:grammar} is based on that extension.

\subsubsection{Linear Temporal Logic}\label{sec:ltl}
Property specifications in the linear-time perspective are called LTL formulae and are usually denoted by $\phi$. LTL formulae are constructed from atomic propositions $a$, the boolean connectors conjunction ($\land$) and negation ($\lnot$), and the temporal modalities \emph{next} (denoted by $\X$) and \emph{until} (denoted by $\U$). The syntax of an LTL formula $\phi$ is described by the grammar:

\begin{table}[!ht]
    \centering
    $
    \begin{array}{rcllllllllllll}
        \phi &::= &true &| &a &| &\phi_1 \land \phi_2 &| &\lnot \phi &| &\X\phi &| &\phi_1 \U \phi_2
    \end{array}
    $
    \caption{Grammar for Linear Temporal Logic formulae}
    \label{tab:ltl-grammar}
\end{table}

where $a \in AP$ and $AP$ is a set of atomic propositions. It should be noted that the grammar in \autoref{tab:ltl-grammar} is not in Backus Naur Form but is rather an \emph{abstract} syntax~\cite{baier2008principles}. The remaining propositional operators not present in the grammar are omitted as they can be derived from conjunction and negation. Similarly the temporal operators \emph{eventually} and \emph{always} are omitted as they can be derived from the \emph{until} operator using the following rules~\cite[pp.~232]{baier2008principles}:
\begin{align}
    &\F \phi = true \U \phi \label{eq:eventually} \\
    &\G \phi = \lnot \F \lnot \phi \label{eq:always}
\end{align}

In the given grammar the operator precedence is that unary operators bind stronger than the binary ones. \emph{Until} (binary) binds stronger than the binary operators from propositional logic ($\land$, $\lor$ and $\imply$). Furthermore the \emph{until} operator is right-associative, i.e. $\phi_1 \U \phi_2 \U \phi_3 \equiv \phi_1 \U \left(\phi_2 \U \phi_3\right)$~\cite{baier2008principles}.

The length of an LTL formula $\phi$ is denoted by $| \phi |$ and is determined by the number of operators in $\phi$, being both propositional and temporal operators~\cite{baier2008principles}. For instance, let $\phi= true$ or $\phi= a$ then $| \phi | = 0$. Let $\phi= true \U \left( \lnot a \land \X b \right)$ then $| \phi | = 4$. The length of formulae is used for proving our algorithm for determining elementary sets in \autoref{sec:methods-dt}.

LTL formulae are said to either hold under a path or not. To determine whether an LTL formula $\phi$ holds under a path or not the formula is defined as a language $\mathit{Words}(\phi)$ which is all infinite words over $2^{AP}$ that satisfies $\phi$. $\mathit{Words}(\phi)$ can be described more formally as~\cite[Def.~5.6]{baier2008principles}
\begin{align*}
    \mathit{Words}(\phi) = \left\{ \sigma \in \left( 2^{AP} \right)^\omega \mid \sigma \models \phi \right\}
\end{align*}
For a word $\sigma = A_0 A_1 A_2 \ldots \in \left( 2^{AP} \right)^\omega$ the suffix of $\sigma$ starting at the $i$\textsuperscript{th} symbol is denoted by $\sigma \left[ i \ldots \right] = A_i A_{i+1} A_{i+2} \ldots$. Given the definition of the words that satisfy an LTL formula, the semantics of the satisfaction relation $\models$ for the grammar of an LTL formula and the derived operators $\F$ and $\G$ can be listed~\cite[Fig.~5.2]{baier2008principles}:
\begin{align}
    \sigma &\models true                && \label{eq:sem-true}\\
    \sigma &\models a                   &\text{iff }& a \in A_0 \\
    \sigma &\models \phi_1 \land \phi_2 &\text{iff }& \sigma \models \phi_1 \text{ and } \sigma \models \phi_2 \\
    \sigma &\models \lnot \phi          &\text{iff }& \sigma \not\models \phi\\
    \sigma &\models \X \phi             &\text{iff }& \sigma\left[1\ldots\right] = A_1 A_2 A_3\ldots \models \phi\\
    \sigma &\models \phi_1 \U \phi_2    &\text{iff }& \exists j \geq 0.\enskip \sigma\left[j\ldots\right] \models \phi_2 \text{ and } \sigma\left[i\ldots\right] \models \phi_1, \text{ for all } 0 \leq i < j \\
    \sigma &\models \F \phi             &\text{iff }& \exists i \geq 0.\enskip \sigma\left[i\ldots\right] \models \phi \label{eq:sem-eventually} \\ 
    \sigma &\models \G \phi             &\text{iff }& \forall i \geq 0.\enskip \sigma\left[i\ldots\right] \models \phi \label{eq:sem-always}
\end{align}

As the grammar of LTL formulae might suggest, see \autoref{tab:ltl-grammar}, it is possible to create composite operators like
\begin{align*}
    &\G\F\phi \quad \text{``always eventually'' or ``infinitely often''}\\
    &\F\G\phi \quad \text{``eventually always'' or ``eventually forever''}
\end{align*}
The informal semantics of $\G\F\phi$ is that it holds under a word if and only if every suffix of the word contains a symbol for which $\phi$ holds under. Furthermore $\F\G\phi$ holds under a word if and only if there exists a suffix for which $\phi$ holds under every suffix of the suffix~\cite{baier2008principles}. The formal semantics can be derived from \autoref{eq:sem-eventually} and \autoref{eq:sem-always}.

\subsection{Closure of $\phi$}
The closure of an LTL formula $\phi$ is the set containing all subformulae $\psi$ of $\phi$ and their negation $\lnot \psi$ together with the formula itself and its negation~\cite[Def.~5.34]{baier2008principles}. Consider the following example
\begin{align*}
    \phi &= true \U \left( a \land \X b \right) \\
    closure(\phi) &= \left\{ \phi, \lnot \phi, a \land \X b, \lnot \left( a \land \X b \right), \X b, \lnot \X b, a, \lnot a, b, \lnot b, true \right\}
\end{align*}
For readability we omit to write $\lnot true$ in $closure(\phi)$ as a set containing $\lnot true$ can not be elementary, as will be shown in the following section. The length of $closure(\phi)$ is denoted by $|closure(\phi)|$ and $|closure(\phi)| \in \mathcal{O}\left(|\phi|\right)$~\cite{baier2008principles}. The concept of closure is introduced for constructing the elementary sets of LTL formulae and thus is a crucial part of the automata-based model checking method used.

\subsection{Elementary Sets}\label{sec:elemesets}
A set $B \subseteq closure(\phi)$ is said to be \emph{elementary} if it satisfies the following properties:

The first set of properties guarantee that the set $B$ is \emph{consistent} with respect to propositional logic. That is for all conjunctions $\phi_1 \land \phi_2 \in closure(\phi)$ and subformulae $\psi \in closure(\phi)$ the following properties should be satisfied~\cite[Fig.~5.20]{baier2008principles}:
\begin{align}
    \phi_1 \land \phi_2 \in B &\Leftrightarrow \phi_1 \in B \text{ and } \phi_2 \in B \label{eq:elem1.1} \\
    \psi \in B &\imply \lnot \psi \notin B \label{eq:elem1.2} \\
    true \in closure(\phi) &\imply true \in B \label{eq:elem1.3}
\end{align}

The second set of properties guarantee that the set $B$ is \emph{locally consistent} with respect to the temporal operator \emph{until}. That is for all $\phi_1 \U \phi_2 \in closure(\phi)$ the following properties should be satisfied~\cite[Fig.~5.20]{baier2008principles}:
\begin{align}
    \phi_2 \in B &\imply \phi_1 \U \phi_2 \in B \label{eq:elem2.1} \\
    \phi_1 \U \phi_2 \in B \text{ and } \phi_2 \notin B &\imply \phi_1 \in B \label{eq:elem2.2}
\end{align}

The third and final set of properties is that the set $B$ is \emph{maximal}, i.e. all subformula $\psi$ of $closure(\phi)$ should either be in $B$ or not be in $B$. That is for all $\psi \in closure(\phi)$ the following property should be satisfied~\cite[Fig.~5.20]{baier2008principles}:
\begin{align}
    \psi \notin B \imply \lnot \psi \in B \label{eq:elem3.1}
\end{align}

Elementary sets is a fundamental part of the verification of LTL properties or, more explicitly, in the conversion from LTL formulae to generalised nondeterministic Büchi automata. Finally let us consider a concrete example:

\begin{example}
Consider the following LTL formula and its closure:
\begin{align*}
    \phi &\eqdef a \U \left( a \land b \right) \\
    closure(\phi) &= \left\{ a \U \left( a \land b \right), \lnot \left( a \U \left( a \land b \right) \right), a \land b, \lnot \left( a \land b \right), a, \lnot a, b, \lnot b \right\}
\end{align*}
Below all sets $B \subseteq closure(\phi)$ are listed that satisfy property (\ref{eq:elem3.1}). Each set is marked whether or not it is elementary. In case a set is not marked as elementary, the property that was violated by the set will be identified. Only maximal subsets are considered as it is trivial to see that non-maximal subsets can not be elementary according to property~(\ref{eq:elem3.1}).
\begin{align*}
    B_1    &= \left\{ a,       b,       a\land b,         \phi       \right\} &\text{Elementary} \\
    B_2    &= \left\{ \lnot a, b,       a\land b,         \phi       \right\} &\text{Violates property (\ref{eq:elem1.1})} \\
    B_3    &= \left\{ a,       \lnot b, a\land b,         \phi       \right\} &\text{Violates property (\ref{eq:elem1.1})} \\
    B_4    &= \left\{ \lnot a, \lnot b, a\land b,         \phi       \right\} &\text{Violates property (\ref{eq:elem1.1})} \\
    B_5    &= \left\{ a,       b,       \lnot (a\land b), \phi       \right\} &\text{Violates property (\ref{eq:elem1.1})} \\
    B_6    &= \left\{ \lnot a, b,       \lnot (a\land b), \phi       \right\} &\text{Violates property (\ref{eq:elem1.1}) \& (\ref{eq:elem2.2})} \\
    B_7    &= \left\{ a,       \lnot b, \lnot (a\land b), \phi       \right\} &\text{Elementary} \\
    B_8    &= \left\{ \lnot a, \lnot b, \lnot (a\land b), \phi       \right\} &\text{Violates property (\ref{eq:elem2.2})} \\
    B_9    &= \left\{ a,       b,       a\land b,         \lnot \phi \right\} &\text{Violates property (\ref{eq:elem2.1})} \\
    B_{10} &= \left\{ \lnot a, b,       a\land b,         \lnot \phi \right\} &\text{Violates property (\ref{eq:elem1.1}) \& (\ref{eq:elem2.1})} \\
    B_{11} &= \left\{ a,       \lnot b, a\land b,         \lnot \phi \right\} &\text{Violates property (\ref{eq:elem1.1}) \& (\ref{eq:elem2.1})} \\
    B_{12} &= \left\{ \lnot a, \lnot b, a\land b,         \lnot \phi \right\} &\text{Violates property (\ref{eq:elem1.1}) \& (\ref{eq:elem2.1})} \\
    B_{13} &= \left\{ a,       b,       \lnot (a\land b), \lnot \phi \right\} &\text{Violates property (\ref{eq:elem1.1})} \\
    B_{14} &= \left\{ \lnot a, b,       \lnot (a\land b), \lnot \phi \right\} &\text{Elementary} \\
    B_{15} &= \left\{ a,       \lnot b, \lnot (a\land b), \lnot \phi \right\} &\text{Elementary} \\
    B_{16} &= \left\{ \lnot a, \lnot b, \lnot (a\land b), \lnot \phi \right\} &\text{Elementary} 
\end{align*}
Thus the sets $B_1$, $B_7$, $B_{14}$, $B_{15}$ and $B_{16}$ are elementary.
\end{example}