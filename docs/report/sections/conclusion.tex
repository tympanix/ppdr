\section{Conclusion}
In this paper, we have defined a graph-based data repository abstraction of data resources, that can be transformed into an equivalent transition system. The transformation enables the possibility of formulating LTL formulae based on the existing abstract grammar presented by (Baier and Katoen, 2008)\cite{baier2008principles} and verify that the formulae hold under the data repository for some data resource by using the automata-based model checking algorithm~\cite{baier2008principles}. In order to allow for more complex formulae and context-aware formulae, we presented a new grammar for constructing policy formulae, that is based on the existing abstract grammar of LTL~\cite{baier2008principles}. The policy formulae can represent either confidentiality formulae or integrity formulae. To use the existing automata-based model checking algorithm for verifying the policy formulae, we present a transformation from policy formulae following the new grammar to a corresponding LTL formula. To ensure that the properties defined by the confidentiality policies are provenance preserving, dependencies between data resources are created and inheritance of confidentiality policies are enforced.

With a proof of concept implementation written in Go, we have shown that the theoretical work that has been presented in the paper can be practiced~\cite{}. A data repository of data resources with dependencies to each other can be created, confidentiality policies can be associated with resources and are enforced when inherited. The improved algorithm for finding elementary sets using a decision tree approach has also been implemented and integrated into the automata-based model checking algorithm. Furthermore is the implementation able to handle pure LTL formulae and thus offers an implementation in Go for verifying LTL formulae as well.

Several lines of future works to consider have been identified. The future works are theoretical and practical and are concerned with optimizing the time and space complexity of the model checking algorithm used and making the grammar for policy formulae more expressive. The identified suggestions are likely to be considered before the implementation can be deployed on a production scale. As we have defined how to transform PF to LTL and DR to TS, it would be possible to consider alternative algorithms for LTL model checking. An example could be the approach (Wulf et al., 2008)\cite{de2008antichains} suggest, which uses alternating automata with exploration techniques based on antichains. Another would be (Gerth et al., 1995)\cite{gerth1995simple}, which presents a simple on-the-fly algorithm for LTL verification.
