\section{Conclusion}
In this paper, we have defined a graph-based data repository abstraction of data resources, that can be transformed into an equivalent transition system. The transformation enables the possibility of formulating LTL formulae based on the existing abstract grammar presented by~\cite{baier2008principles} and verify that the formulae are satisfied by the data repository for some data resource by using the automata-based model-checking algorithm~\cite{baier2008principles}. In order to allow for more complex formulae and context aware formulae, we presented a new grammar for constructing policy formulae, that extends the existing abstract grammar of LTL~\cite{baier2008principles}. The policy formulae can represent either a confidentiality formulae, which impose constraints about the subjects accessing the data resource, or a integrity formulae, which impose constraints about the quality of the data resource. To use the existing automata-based model-checking algorithm for verifying the satisfiability of the policy formulae, we present a transformation from a policy formulae following the new grammar to an equivalent LTL formulae. To ensure that the properties defined by the confidentiality policies are provenance preserving are dependencies between data resources created and inheritance of confidentiality policies are enforced.

With a proof of concept implementation written in Go, we have shown that the theoretical work that have been presented in the paper can be practiced~\cite{}. A data repository of data resources with dependencies to each other can be created, confidentiality policies can be associate to resources and are enforce when inherited. The improved algorithm for finding elementary sets using a decision tree approach have also been implemented and integrated in the automata-based model-checking algorithm. Furthermore is the implementation able to handle pure LTL formulae and thus offers an implementation in Go for checking the satisfiability of LTL formulae as well.

Several lines of future works to consider have been identified. The future works are theoretical and practical and are concerned with optimising the time and space complexity of the model-checking algorithm used. The identified suggestions are likely to be considered before the implementation can be deployed on a production scale.