\section{Related Works}
The problem this paper addresses requires that a property is verified against a transition system. Multiple approaches can be used to check if a transition system satisfy the properties, when the properties are defined by LTL formulae.

One of these approaches is implemented by \cite{serejke}. The solution presents a command line interface (CLI) tool, that takes a LTL formula and a transition system as arguments and outputs \textit{yes} or \textit{no} corresponding to whether the transition system satisfied the LTL formula or not. Furthermore in case of \textit{no} is a counter example provided. Whereas the tool is fully functional as a standalone application, it is not easily integrated in other application as it all interaction is through the CLI and no API is available. Futhermore is the tool limited to two primitive type, namely being \textit{int} and \textit{bool}. Explicit assignment of a condition of a transition is also required as well as an assignment.

Another approach is presented by \cite{erlkoenig90}, which for LTL formulae calculates the closure, valid atoms and generates a tableau. This tool utilises a CLI as well, which complicates integration of the tool in other applications. Furthermore to integrate this tool will require to parse the contents of the tableau, which is written to a PDF.

The tool implemented by \cite{jpsember} performs model checking on LTL formulae and is meant to be used as an instruction aid for the study of temporal logics. As might be implied, the tool utilises CLI and as previous presented tools is not suitable to be integrated in other solutions. The LTL formulae are checked by transforming the negation of the formulae themselves and the model of the system, which is described as a Kripke model, to equivalent Büchi automatons. The product of the two automatons is then created and checked if it accepts infinite strings.