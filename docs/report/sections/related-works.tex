\section{Related Works}\label{sec:related-works}
The basic underlying problem in this paper is to check whether a property formula is verified against a transition system. Multiple approaches can be used to check if the properties hold in a transition system when the properties are defined by LTL formulae.

\paragraph{Model checkers}
A popular and widely used implementation of a model checking system is the SPIN verification tool, which is used to verify multi-threaded software~\cite{spin}. SPIN solves related problems of model checking which are presented in this paper, in particular verification of properties expressed as LTL formulae.

Another model checker is NuSMV, which is a symbolic model extended from SMV, the first model checker based on Binary Decision Diagrams (BDD)~\cite{nusmv}. NuSMV supports properties defined in both LTL and CTL. Furthermore, NuSMV has been designed to be reliably used for verifying industrially scaled designs and to be used as a core in custom verification tools~\cite{nusmv}.

A last honorable mention is the UPPAAL verification system for modelling of real-time systems~\cite{uppaal}. UPPAAL works with networks of timed automata and implements a model checking algorithm for a formal language closely inspired by LTL.

\paragraph{Path query languages}
(Hellings et al., 2013)\cite{hellings2013walk} have investigated the Walk Logic (WL), which is an extension of FOL with explicit quantifiers over the walks in a graph. They propose WL, a language which is not considered user-friendly, as a framework for investigating the expressive power of path query languages for graph databases~\cite{hellings2013walk}. WL is a more expressive logic compared to LTL however, they leave a number of open questions like: ``what is the precise data complexity of WL?'', and ``is hybrid CTL$^\ast$ strictly more expressive than other regular walk logics?''~\cite{hellings2013walk}.

Regular expressions is a well-known method for parsing sequences of characters. When paths in a graph is represented as a strings, one can express properties of these paths using regular expressions which makes it a powerful path query language. 

Another language is that of XQuery\footnote{\href{https://www.w3.org/TR/xquery/all/}{https://www.w3.org/TR/xquery/all/}}, originally designed to express properties of structured XML. Since XML documents can be presented as a graph of XML nodes, this makes XQuery a path query language for the domain of XML. One could, for example, serialize an arbitrary graph into XML and then utilize XQuery to express properties of that graph.

\paragraph{Access control systems}
The work presented in this paper can be perceived as a type of access control system orthogonal to access control systems like Role-Based Access Control (RBAC) or Access Control Lists (ACL), which takes into account the flow of information. Although this paper presents a different approach to access control systems, one could utilize the work presented in this paper to simulate both an RBAC system or an ACL system. Therefore we present an abstract approach to access control systems, which can take many forms varying from the specifics on each use case, but in general, is more dynamic than what is conventionally found in RBAC and ACL systems.

Another modern approach is the Decentralized Labelling Model (DLM), which is concerned with \emph{``controlling information flow in systems with mutual distrust and decentralized authority''}~\cite{myers1997decentralized}. In DLM each individual defines security policies for the propagation of data with labels annotating the data, contrary to a central authority. Users of the system can manipulate data using programs while keeping information in an enclosed environment. Only when the data leaves the internal environment, the data can be considered as leaked.

\paragraph{Frameworks} (Lafuente, 2019)\cite{lafuente2019framework} presents reductions on provenance-preserving histories in directed acyclic graphs which is compatible with an arbitrary provenance language of choice. As such, (Lafuente, 2019)\cite{lafuente2019framework} focuses on reduction on provenience histories in an abstract domain, where in this paper, we focus on a concrete realization of the provenance language and the acyclic data graph. Since (Lafuente, 2019)\cite{lafuente2019framework} treat provenance information in an abstract manner, the work could apply to the concrete solutions presented in this paper.