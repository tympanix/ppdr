%\RequirePackage[12tabu, orthodox]{nag}

\documentclass[a4paper]{article}
\usepackage[table,xcdraw]{xcolor}
\usepackage[english]{babel}
\usepackage[a4paper,includeheadfoot,margin=4.cm]{geometry}
\usepackage[utf8]{inputenc}
\usepackage{textcomp}
\usepackage[toc,page]{appendix}
\usepackage{amsmath}
\usepackage{mathtools}
\usepackage{wasysym}
\usepackage{multirow}
\usepackage{amssymb}
\usepackage{nameref}
\usepackage{url}
\usepackage{graphicx}
\usepackage{caption}
\usepackage{subcaption}
\usepackage{tikz-qtree}
\usepackage{listings}
\usepackage{color}
\usepackage{microtype}
\usepackage{siunitx}
\usepackage{booktabs}
\usepackage{syntax}
\setlength{\grammarindent}{4em}
\usepackage[ampersand]{easylist}
%\usepackage[total={6.7in,10.2in}, % text area width and height
%		    top=1in, left=1in, right=1in, bottom=1in,% top and left margin
%		    includefoot]{geometry}
\usepackage[colorlinks=false, pdfborder={0 0 0}]{hyperref}
\usepackage{cleveref}
\usepackage[nottoc,numbib]{tocbibind}
\usepackage{pgf}
\usepackage{tikz}
\usepackage{mathrsfs}
\usepackage{amsthm}
\usepackage{lipsum}
\usepackage{forest}
\usepackage{algorithm2e}
\usepackage{mdframed}
\RestyleAlgo{boxed}

\usetikzlibrary{automata,arrows.meta,positioning}

\tikzstyle{tikz-ts}=[every loop/.style={looseness=5}]

\theoremstyle{definition}
\newtheorem{example}{Example}[section]
\newcommand{\exampleautorefname}{Example}

\theoremstyle{definition}
\newtheorem{definition}{Definition}[section]
\newcommand{\definitionautorefname}{Definition}

\theoremstyle{definition}
\newtheorem{lemma}{Lemma}[section]
\newcommand{\lemmaautorefname}{Lemma}

% Thick centred dot
\newcommand{\tcdot}{\raisebox{.4ex}{\tiny $\bullet$}}

\newcommand\eqdef{\mathrel{\stackrel{\makebox[0pt]{\mbox{\normalfont\tiny def}}}{=}}}

% Temporal operators
\newcommand{\X}{\bigcirc} % next
\newcommand{\G}{\square} % always
\newcommand{\F}{\Diamond} % eventually
\newcommand{\U}{\cup} % until

% Propositional logic operators
\newcommand{\imply}{\rightarrow} % implication
\newcommand{\biimply}{\leftrightarrow} % bi-implication

\newcommand{\conf}{\chi}
\newcommand{\inte}{\iota}
\newcommand{\pf}{\alpha}
\renewcommand{\phi}{\varphi} % phi
\newcommand{\Tau}{\mathcal{T}}

%Code for DTU logo as background picture.
\usepackage{eso-pic}
\newcommand\BackgroundPic{%
\put(10,5){%
\parbox[b][13cm]{32cm}{%
\vfill
\centering
\includegraphics[width=6cm,height=\paperheight,%
keepaspectratio]{./input/DTU.jpg}%
\vfill
}}}

\definecolor{codegreen}{rgb}{0,0.6,0}
\definecolor{codegray}{rgb}{0.5,0.5,0.5}
\definecolor{codepurple}{rgb}{0.85,0,0.85}
\definecolor{backcolour}{rgb}{0.95,0.95,0.92}

\lstdefinestyle{mystyle}{
    backgroundcolor=\color{backcolour},
    commentstyle=\color{codegreen},
    keywordstyle=\color{blue},
    numberstyle=\tiny\color{codegray},
    stringstyle=\color{codepurple},
    basicstyle=\footnotesize,
    breakatwhitespace=false,
    breaklines=true,
    captionpos=b,
    keepspaces=true,
    numbers=left,
    numbersep=5pt,
    showspaces=false,
    showstringspaces=false,
    showtabs=false,
    tabsize=2
}

\lstset{style=mystyle}
