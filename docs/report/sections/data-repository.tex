\section{Data Repository}\label{sec:data-repository}
% How is the data repository modelled.
When thinking of what a data repository is, words like ``tuple spaces'', ``databases'' or even ``comma-separated files'' might spring to mind. However, to utilize LTL formulae to specify properties about the data and their provenance in the repository, it is necessary to carefully consider what data repository abstraction to use. 

% Paragraph:
% - Make the connection to transition systems
In the literature (Baier and Katoen, 2008)\cite{baier2008principles} use a Transition System (TS) without terminal states as a reactive system, for which they are checking LTL formulae against. The following definition of a TS is used:
\begin{definition}[Transition System (TS)~{\cite[Def.~2.1]{baier2008principles}}]\label{def:ts}
A \emph{transition system} $TS$ is a tuple $\left(S, \longrightarrow, I, AP, L \right)$, where
\begin{itemize}
  \item $S$ is a set of data states
  \item $\longrightarrow \subseteq S \times S$ is a transition relation
  \item $I \subseteq S$ is a set of initial states
  \item $AP$ is a set of atomic propositions
  \item $L : S \rightarrow 2^{AP}$ is a labelling function
\end{itemize}
Note that the actions on transition labels have been omitted as they are not relevant in this setting.
\end{definition}

To model a data repository as a TS, one could consider data resources from the data repository as states in the TS and dependencies between data resources as transitions between states. Let $r_i$ and $r_j$ be data resources in the same data repository $DR$, and let $r_i$ be derived fully or partially from $r_j$, i.e. information from $r_j$ was used when creating $r_i$. Then there exists a dependency from $r_i$ to $r_j$. As an example consider a data repository of scientific papers, i.e. each paper is a data resource. Say this paper is the data resource $r$ in the data repository. Then $r$ would have a dependency to each of the papers it references, as the information from those references was used when composing this paper. It could also be that a dependency exists to a resource that defines rules about the format of the paper. This is what is referred to as a policy and will be introduced in a later section. Modeling dependencies in this way inherently describes the past and if you were to follow the dependencies from a data resource they would paint a picture of how that resource got to be. This is contrary to transitions in a TS, which describes something about the future.

% Paragraph: 
% - No terminal states
As mentioned (Baier and Katoen, 2008)\cite{baier2008principles} use a TS without any terminal states to ensure that all paths and traces in the TS are infinite. To achieve infinite paths and thereby infinite traces in the data repository modeled as a TS, all data resources without any dependencies to other data resources have a dependency to itself, a so-called \emph{self-loop}. A self-loop implies that the data resource was created independently from any other data resource. By introducing self-loops one ensures that no terminal state can ever be reached or will even exist in the data repository, and as a result all paths in the data repository are infinite.

% Paragraph: 
% - AP
% - Labelling function
In a TS a set of atomic propositions is related to a state by the labeling function. On the other hand, in a data repository, it is desired to have attribute names and values being mapped to a data resource. Attribute names are simply names that map to a value that is given by a literal.

% Paragraph: 
% - Initial states
According to \autoref{def:ts} the initial states of a TS should be explicitly defined. However, it is not desirable that the data repository itself should restrict, which data resources can and can not be read, i.e. be the initial state in the data repository. Therefore there is no need for the data repository to keep track of the initial data resources.

Whereas a TS does not store the LTL formulae themselves, it is desired to keep track of the policy formulae, which are introduced in \autoref{sec:grammar}, as well as being able to map a data resource to a policy formula.

% Summarize the definition of a data repository
Given all of the above descriptions we define a data repository as follows:
\begin{definition}[Data Repository (DR)]
A \emph{data repository} $DR$ is a tuple $\DR$, where
\begin{itemize}
  \item $R$ is a set of data resources
  \item $\longrightarrow \subseteq R \times R$ is a dependency relation
  \item $A$ is a finite set of attribute names
  \item $L$ is a set of literals
  \item $M : R \times A \rightharpoonup L$ is a partial mapping from data resources and attributes to literals
  \item $F$ is a set of policy formulae
  \item $P : R \rightarrow F$ is a mapping from data resources to policy formulae
\end{itemize}
\end{definition}

\begin{example}[Data repository]\label{ex:data-repo}
Let us consider how a data repository $\DR$ with five data resources, that is $R = \left\{ r_1, r_2, r_3, r_4, r_4 \right\}$, could look like. \autoref{fig:data-repo} shows how the five data resources could depend on each other, i.e. how $\longrightarrow$ is defined. The data resource $r_5$ has dependencies to $r_3$ and $r_4$, $r_3$ has a dependency to $r_1$ and $r_4$ has a dependency to $r_2$. The data resources $r_1$ and $r_2$ was created completely independently and have no dependency to other resources, thus they have a \emph{self-loop}.
\begin{figure}[!ht] 
    \begin{center}
        \begin{forest}
    for tree={circle, draw, s sep=50pt, grow=north,edge={->}}
    [$r_5$, name=r5, tikz={\node[draw=none, inner sep=0pt, right=2pt of .east]  {\{$a_1, a_2, a_3, a_4, a_4$\}};}
        [$r_4$, name=r4, tikz={\node[draw=none, inner sep=0pt, right=2pt of .east]  {\{$a_2, a_4$\}};}
            [$r_2$, name=r2, tikz={\node[draw=none, inner sep=0pt, right=2pt of .east]  {\{$a_2$\}};}]
        ]
        [$r_3$, name=r3, tikz={\node[draw=none, inner sep=0pt, right=2pt of .east]  {\{$a_1, a_3$\}};}
            [$r_1$, name=r1, tikz={\node[draw=none, inner sep=0pt, right=2pt of .east] {\{$a_1$\}};}]
        ] 
    ]
    \draw[->] (r1) to [out=north west,in=north east,looseness=4] (r1);
    \draw[->] (r2) to [out=north west,in=north east,looseness=4] (r2);
\end{forest}
        \caption{Sample data repository containing five data resources, their dependencies, attribute names and literals.}
        \label{fig:data-repo}
    \end{center}
\end{figure}
The set of attribute names $A$ and the set of literals $L$ are defined as follows
\begin{align*}
    A &= \left\{ a_1, a_2, a_3, a_4, a_4 \right\} \\
    L &= \left\{ l_1, l_2, l_3, l_4, l_4 \right\}
\end{align*}
From \autoref{fig:data-repo} the mapping of a data resource and an attribute name to a literal can be seen, e.g. $M(r_1, a_1) = l_1$. Note the set of policy formulae $F$ and the mapping from data resources to policy formulae $P$ has been omitted in this example as it will be introduced later in \autoref{sec:policies}.
\end{example}

\subsection{Context}
A context is introduced, where the main purpose of the context is to populate values that can be used in resolving policy formulae that follow the grammar in \autoref{sec:grammar}. The context is defined as follows:
\begin{definition}[Context]
A \emph{context} $c$ is a tuple $\CON$, where
\begin{itemize}
    \item $s$ is a $user$ literal corresponding to the current subject
    \item $U$ is a set of $usr$ literals
    \item $I$ is a set of $str$ literals
    \item $N : I \rightarrow U$ is a mapping from user identities to user literals
\end{itemize}
\end{definition}
The primary value of concern is the current subject $s$, i.e. the name of the user/person performing operations against the data repository. The implementation abstracts away from authenticating the user, i.e. ensuring that the user is whom he/she claims to be. However, it should be trivial to see that an authentication system could be placed as a layer on top of the work presented in this paper.

In theory, there is no limit to what the context could be extended with, e.g. the system that the operation is performed from, the name of the system, etc.

% Content:
% - Explain how it extends the grammar of LTL
% - Only some propositional logic are directly implemented but the remaining can be used indirectly
% - Explain that the grammar is abstract and somethings are not explicitly mentioned
\subsection{Grammar}\label{sec:grammar}
In this section a new type of formula is introduced named Policy Formula ($PF$) which is inspired by LTL, which was introduced in \autoref{sec:ltl}. This type of formula is used to formulate confidentiality and integrity policies in a data repository. We distinct between two types of representations of Policy Formulae, each defined by their own grammar, called User Policy Formulae ($UPF$) and Internal Policy Formulae ($IPF$). This abstraction is necessary to distinguish between the formulae as presented externally to the actor of the system and the internal representation which is exposed to the model checker. The grammar for $UPF$ can be seen in \autoref{tab:pf-grammar-user} and for $IPF$ in \autoref{tab:pf-grammar}.

\begin{table}[!ht]
    \centering
    $
    \begin{array}{rcll}
        \pf_u   & ::=   & true              & \text{(true predicate)} \\
                & |     & \pfn              & \text{(extension)} \\
                & |     & \pf_{u_1} \land \pf_{u_2} & \text{(conjunction)} \\
                & |     & \pf_{u_1} \lor \pf_{u_2}  & \text{(disjunction)} \\
                & |     & \pf_{u_1} \imply \pf_{u_2} & \text{(implication)} \\
                & |     & \neg \pf_u          & \text{(negation)} \\
                & |     & \X \pf_u            & \text{(next)} \\
                & |     & \pf_{u_1} \U \pf_{u_2}    & \text{(until)} \\
                & |     & \F \pf_u            & \text{(eventually)} \\
                & |     & \G \pf_u            & \text{(always)} \\
        \pfn    & ::=   & e_1 \bowtie e_2   & \text{(expression)} \\
                & |     & ap                & \text{(atomic proposition)} \\
                & |     & self              & \text{(self predicate)} \\
        f       & ::=   & user(str)           & \text{(user function)} \\
                & |     & subject()         & \text{(subject function)} \\
        e       & ::=   & atr               & \text{(attribute name)} \\
                & |     & f                 & \text{(functions)} \\
                & |     & l                 & \text{(literal)} \\
        \bowtie & ::=   & =                 & \text{(equal)} \\
                & |     & \neq              & \text{(not equal)} \\
                & |     & <                 & \text{(less than)} \\
                & |     & \leq              & \text{(less than equal)} \\
                & |     & >                 & \text{(greater than)} \\
                & |     & \geq              & \text{(greater than equal)} \\
        l       & ::=   & str               & \text{(string literal)} \\
                & |     & num               & \text{(number literal)} \\
                & |     & bol               & \text{(boolean literal)} \\
    \end{array}
    $
    \caption{Grammar for user policy formula}
    \label{tab:pf-grammar-user}
\end{table}

\begin{table}[!ht]
    \centering
    $
    \begin{array}{rcll}
        \pf     & ::=   & true              & \text{(true predicate)} \\
                & |     & \pfn              & \text{(extension)} \\
                & |     & \pf_1 \land \pf_2 & \text{(conjunction)} \\
                & |     & \neg \pf          & \text{(negation)} \\
                & |     & \X \pf            & \text{(next)} \\
                & |     & \pf_1 \U \pf_2    & \text{(until)} \\
        \pfn    & ::=   & e_1 \bowtie e_2   & \text{(expression)} \\
        e       & ::=   & atr               & \text{(attribute name)} \\
                & |     & l                 & \text{(literal)} \\
        \bowtie & ::=   & =                 & \text{(equal)} \\
                & |     & \neq              & \text{(not equal)} \\
                & |     & <                 & \text{(less than)} \\
                & |     & \leq              & \text{(less than equal)} \\
                & |     & >                 & \text{(greater than)} \\
                & |     & \geq              & \text{(greater than equal)} \\
        l       & ::=   & str               & \text{(string literal)} \\
                & |     & num               & \text{(number literal)} \\
                & |     & bol               & \text{(boolean literal)} \\
                & |     & usr               & \text{(user literal)} \\
                & |     & rsc               & \text{(resource literal)} \\
                & |     & nil               & \text{(nil literal)}
    \end{array}
    $
    \caption{Grammar for internal policy formula}
    \label{tab:pf-grammar}
\end{table}

To couple the two types of Policy Formulae we define a translation from $UPF$ to $IPF$. This transformation replaces syntax from the grammar of $UPF$ with equivalent grammar of $IPF$ to produce a more concise representation. These replacements include \emph{syntactic sugar}, i.e. syntax which is exposed to the actor of the system but can be expressed with existing syntax without loss of semantics. Specifically for $IPF$ we introduce user literals ($usr$) and resource literals ($rcs$), which are not allowed in $UPF$, i.e. manipulation and construction of these literals are only allowed in the internal representation. It should be noted that the resource literal ($rcs$) can be expressed equivalently with a data resource $r \in R$ of a data repository $\DR$. In $UPF$ one resolves to the $subject()$ function, which resolves the user literal for the current subject performing an action on the data repository.

\begin{definition}[Transformation from UPF to IPF]\label{def:pf-user-to-internal}
Given a data resource $r$ and an execution context $\CON$ the UPD $\upf$ can be transformed to an IPF $\pf$, denoted $\pf = [\pf_u]_{c,r}$ defined by the following:
\begin{itemize}
    \item $[true]_{c,r} \eqdef true$
    \item $[\pf_{u_1} \land \pf_{u_2}]_{c,r} \eqdef [\pf_{u_1}]_{c,r} \land [\pf_{u_2}]_{c,r}$
    \item $[\pf_{u_1} \lor \pf_{u_2}]_{c,r} \eqdef \lnot(\lnot [\pf_{u_1}]_{c,r} \land \lnot [\pf_{u_1}]_{c,r})$
    \item $[\pf_{u_1} \imply \pf_{u_2}]_{c,r} \eqdef \lnot([\pf_{u_1}]_{c,r} \land \lnot [\pf_{u_1}]_{c,r})$
    \item $[\lnot \pf_{u}]_{c,r} \eqdef \lnot [\pf_{u}]_{c,r}$
    \item $[\X \pf_{u}]_{c,r} \eqdef \X [\pf_{u}]_{c,r}$
    \item $[\pf_{u_1} \U \pf_{u_2}]_{c,r} \eqdef [\pf_{u_1}]_{c,r} \U [\pf_{u_2}]_{c,r}$
    \item $[\F \pf_{u}]_{c,r} \eqdef true \U [\pf_{u}]_{c,r}$ \hfill(using \autoref{eq:eventually})
    \item $[\G \pf_{u}]_{c,r} \eqdef \lnot (true \U \lnot [\pf_{u}]_{c,r})$ \hfill(using \autoref{eq:eventually} and \autoref{eq:always})
    \item $[e_1 \bowtie e_2]_{c,r} \eqdef [e_1]_{c,r} \bowtie [e_2]_{c,r}$
    \item $[ap]_{c,r} \eqdef atr = true$ where $atr \eqdef ap$
    \item $[self]_{c,r} \eqdef self = rsc$ where $rsc = r$
    \item $[user(str)]_{c,r} \eqdef
        \begin{cases*}
            N(str)  & iff $str \in I$ \\
            nil     & otherwise
        \end{cases*}$
    \item $[subject()]_{c,r} \eqdef usr$ where $usr = s$
    \item $[atr]_{c,r} \eqdef atr$
    \item $[l]_{c,r} \eqdef l$
\end{itemize}
\end{definition}

As can be seen from \autoref{def:pf-user-to-internal} the transformation is nothing more than a recursive replacement of syntax using the context to resolve operators which are sensitive to the context. Notice that, contrary to LTL, $IPF$ represents atomic propositions as an equality with the boolean literal $true$. Furthermore the $self$ predicate, which has a special meaning in $UPF$, can in the same manner be represented as an equality in $IPF$. Since every predicate of an $IPF$ is of the form $\pfn$, i.e. expressed using some equality of the form $e_1 \bowtie e_2$, the satisfaction relation is defined below for a data resource $r \in R$ belonging to $\DR$:
\begin{align*}
    r &\models e_1 \bowtie e_2 &\text{iff }& eval(e_1, r) [\![ \bowtie ]\!] eval(e_2, r)
\end{align*}
and the evaluation function, $eval : e \times r \rightarrow l$, which is defined by the following:
\begin{align*}
    eval(e,r) =
    \begin{cases*}
        l   & if $e$ is type $atr$ and $\langle r,atr\rangle \mapsto l \in M$ \\
        nil & if $e$ is type $atr$ and $\langle r,atr\rangle \mapsto l \not\in M$ \\
        e         & otherwise (literal)
    \end{cases*}
\end{align*}
By the evaluation of the symbol $e$ we simply distinguish between attribute names, which has be resolved by the mapping $M$ for a given data resource, and literals, which represent themselves. Attributes which are not found in $M$ is evaluated as the $nil$ literal. The attribute names are what allows the actor of the system to reason about any single data resource. Since the mapping $M$ maps to literals, the $\bowtie$ operator is limited to this domain. The semantics for $\bowtie$ is defined following the definitions below. Some types, i.e. numbers, booleans and strings, are defined following the partial ordering of their respective elements.
\begin{align*}
    e_1 [\![ \bowtie ]\!] e_2 \imply& \enskip false \text{ iff } type(e_1) \neq type(e_2) \\
    str [\![ \bowtie ]\!] str' \imply& \enskip \text{lexicographical ordering of strings} \\
    num [\![ \bowtie ]\!] num' \imply& \enskip \text{ordering of numbers in } \mathbb{R} \\
    bol [\![ \bowtie ]\!] bol' \imply& \enskip \text{ordering of booleans in } \mathbb{B} \\
    usr [\![ \bowtie ]\!] usr' \imply&
        \begin{cases*}
            usr = usr'      & if $\bowtie$ is type $=$ (equal) \\
            usr \neq usr'   & if $\bowtie$ is type $\neq$ (not equal) \\
            false       & otherwise
        \end{cases*} \\
    rsc [\![ \bowtie ]\!] rsc' \imply&
        \begin{cases*}
            rsc = rsc'      & if $\bowtie$ is type $=$ (equal) \\
            rsc \neq rsc'   & if $\bowtie$ is type $\neq$ (not equal) \\
            false       & otherwise
        \end{cases*} \\
    nil [\![ \bowtie ]\!] nil' \imply& false
\end{align*}

This concludes the semantic definition of $UPF$ and $IPF$ and how they are evaluated. Lastly we introduce a final transformation from $IPF$ to $LTL$, which allows an $IPF$ formula to be translated and provided to well-known $LTL$ model checking algorithms (these methods algorithms are explored in \autoref{sec:methods}). The approach is to substitute every occurrence of $e_1 \bowtie e_1$ with a new atomic proposition $ap_{e_1 \bowtie e_2}$, which represents the evaluation of the equality. This means that $ap_{e_1 \bowtie e_2}$ is only satisfied for $r \in R$ given $\DR$ iff $r \models e_1 \bowtie e_2$. The transformation from $IPF$ $\pf$ to $LTL$ formula $\phi$ is provided using the following definition:
\begin{definition}[Transformation from $IPF$ to $LTL$]\label{def:pf-to-ltl}
%Given a data repository $\DR$, a data resource $r \in R$ and an $IPF$ $\pf$, the transformation into a $LTL$ formula $\phi$ over the set of atomic propositions $AP_\phi$ and transformation table $\tau : AP_\phi \rightarrow e_1 \bowtie e_2$ is performed by the following. For each $e_1 \bowtie e_2$ in $\pf$ introduce the new atomic proposition and map the atomic proposition in $\tau$:
Given an $IPF$ $\pf$, the transformation into a $LTL$ formula $\phi$ over the set of atomic propositions $AP_\phi$ and transformation table $\tau : AP_\phi \rightarrow e_1 \bowtie e_2$ is performed by the following. For each $e_1 \bowtie e_2$ in $\pf$ introduce the new atomic proposition and map the atomic proposition in $\tau$:
\begin{align*}
    AP_\phi = AP_\phi \cup \left\{ap_{e_1 \bowtie e_2}\right\} \enskip \text{and} \enskip 
    \tau \cup \{ap_{e_1 \bowtie e_2} \mapsto e_1 \bowtie e_2\}
\end{align*}
Then perform the following substitution:
\begin{align*}
    \pf[^{ap_{e_1 \bowtie e_2}}/_{e_1 \bowtie e_2}]
\end{align*}
\end{definition}
The purpose of $\tau$, as described in \autoref{def:pf-to-ltl}, is to provide a mapping from the newly introduced atomic propositions $ap_{e_1 \bowtie e_2}$ to the substituted subformula $e_1 \bowtie e_2$. One can then satisfy $ap_{e_1 \bowtie e_2}$ iff $r \models \tau(ap_{e_1 \bowtie e_2})$.
\subsection{Policies}\label{sec:policies}
The language of PF introduced in \autoref{sec:grammar} allows one to express properties of paths in a data repository. In the following sections, two new concepts for information flow policies will be introduced, namely, confidentiality and integrity policies.

\subsubsection{Confidentiality Policies}
The concept of confidentiality polices is introduced to allow the user of the system to control how information flows in the future. A confidentiality policy is defined as:
\begin{definition}[Confidentiality policy]\label{def:cp}
A confidentiality policy $\conf$ is an UPF $\upf$ where the formulae is constructed from the language described by the grammar in \autoref{tab:pf-grammar-user}.
\end{definition}
It should be noted that like any other LTL formulae, multiple policies can be defined and concatenated with conjunction. As shown later in \autoref{sec:put} a data resource is associated with a confidentiality policy once the resource is placed in the data repository. 

There are no restrictions on who can depend on a data resource, however, a confidentiality policy will impose constraints on who can read the data resource through a query operation. One might wonder why it makes sense to have one data resource depend on another if the confidentiality policy restricts one from querying it. However, this does open up for the possibility to have a data resource with a corresponding confidentiality policy defined, where the policy can be used as a template for dependant resources through inheritance.

% - Inheritance of policies
Before looking into how one can create these constraints and take advantage of them, it is necessary to explain how confidentiality policies are inherited. The dependencies of a data resource imply that it inherits the dependencies' confidentiality policy, which are inherited policies themselves. The inherited policies are concatenated with conjunction to form one composite policy. As the concatenation is done with conjunction, it means that all inherited policies and the policy that is directly associated with the data resource must be satisfied. Furthermore, this results in that the inherited confidentiality policy will always be as or more restrictive, compared to its individual policies. More formally the confidentiality policy inheritance can be defined as:
\begin{definition}[Confidentiality policy inheritance]\label{def:cpi}
Given a data repository $\DR$ and a data resource $r \in R$ the inherited confidentiality policy $\iconf(DR, r)$ is recursively defined as:
\begin{equation*}
    \iconf(DR, r) = P(r) \land \bigwedge\limits_{r \longrightarrow r' \mid r \neq r'} \iconf(DR, r')
\end{equation*}
The inherited confidentiality policy $\iconf(DR, r)$ is only well-defined if the data repository is acyclic. However, self-loops are ignored, i.e. relations of the form $r \longrightarrow r$. This guarantees that $\iconf(DR, r)$ is a finite policy formula.
\end{definition}

Let us look at an example to visualize the inheritance.
\begin{example}[Confidentiality policy inheritance]
Consider the data repository containing five data resources $r_1\ldots r_5$ illustrated in \autoref{fig:policy-inher}. The data resources $r_1$ and $r_2$ do not have any dependencies, $r_3$ has a dependency to $r_1$, $r_4$ has one to $r_2$ and finally $r_5$ has a dependency to both $r_3$ and $r_4$. Each data resource $r_i$ was placed in the data repository with a corresponding confidentiality policy $\conf_i$.
\begin{figure}[!ht]
    \begin{center}
        \begin{forest}
    for tree={circle, draw, s sep=75pt, grow=north,edge={->}}
    [$r_5$, name=r5, tikz={\node[draw=none, inner sep=0pt, right=2pt of .east]  {$P(r_5)=\conf_5$};}
        [$r_4$, name=r4, tikz={\node[draw=none, inner sep=0pt, right=2pt of .east]  {$P(r_4)=\conf_4$};}
            [$r_2$, name=r2, tikz={\node[draw=none, inner sep=0pt, right=2pt of .east]  {$P(r_2)=\conf_2$};}]
        ]
        [$r_3$, name=r3, tikz={\node[draw=none, inner sep=0pt, right=2pt of .east]  {$P(r_3)=\conf_3$};}
            [$r_1$, name=r1, tikz={\node[draw=none, inner sep=0pt, right=2pt of .east] {$P(r_1)=\conf_1$};}]
        ] 
    ]
    \draw[->] (r1) to [out=north west,in=north east,looseness=4] (r1);
    \draw[->] (r2) to [out=north west,in=north east,looseness=4] (r2);
\end{forest}
        \caption{Inheritance of confidentiality policies.}
        \label{fig:policy-inher}
    \end{center}
\end{figure}
Given the data repository $DR$ in \autoref{fig:policy-inher}, the inherited confidentiality policies of $r_1$ and $r_2$ are $\iconf(DR, r_1) = \conf_1$ and $\iconf(DR, r_2) = \conf_2$. This is expected as they have no dependencies other than a self-loop. Considering the resources $r_3$ and $r_4$ we get $\iconf(DR, r_3) = \conf_3 \land \conf_1$ and $\iconf(DR, r_4) = \conf_4 \land \conf_2$, which is expected as they have a dependency to $r_1$ and $r_2$ respectively. Finally $\iconf(DR, r_5) = \conf_5 \land \conf_4 \land \conf_3 \land \conf_2 \land \conf_1$ as every resource can be reached from $r_5$ through its dependencies.
\end{example}

Now that it is clear how confidentiality policies are inherited, it is possible to consider how to create constraints. Creating constraints is achieved by utilizing the temporal and propositional logic operators as well as the newly introduced functions and predicates to the grammar in \autoref{tab:pf-grammar-user}. In the following, a set of examples will be given to illustrate how constraints can be created through confidentiality policies and the versatility of the language. The examples will use the simple data repository shown in \autoref{fig:conf-policy} as reference if needed.

\begin{figure}[!ht]
    \begin{center}
        \begin{forest}
    for tree={circle, draw, s sep=50pt, grow=west,edge={->}}
    [$r_3$, name=r3
        [$r_2$, name=r2
            [$r_1$, name=r1]
        ] 
    ]
    \draw[->] (r1) to [out=north west,in=north east,looseness=4] (r1);
\end{forest}
        \caption{Data repository containing 3 data resources $r_1, r_2, r_3$.}
        \label{fig:conf-policy}
    \end{center}
\end{figure}

\begin{example}[Trust and distrust of authors]\label{ex:conf-reader-constraints}
Let us start by looking at how one could introduce constraints about who can use a data resource, i.e. query the data resources in the repository and read the contents of it. This can be achieved by introducing a blacklist containing the names of those users who are not allowed to query the data resource. Here the \emph{user} and \emph{subject} functions will become useful, where the subject can be considered as the reader. Considering the data repository $DR$ in \autoref{fig:conf-policy} and say that \emph{Alice} placed $r_1$ in $DR$ and she wish that the users \emph{Mallory} and \emph{Monroe} are restricted from querying the data resource. Adding the following confidentiality policy solves this desire:
\begin{align*}
    \conf_{black} \eqdef subject() \neq user(Mallory) \land subject() \neq user(Monroe)
\end{align*}
However, with $\conf_{black}$ every other user is allowed to query the resources, which might not be the desired outcome. Instead, one could take an approach that is commonly used when defining IP tables, blacklist every IP address by default and explicitly define the allowed ones, i.e. whitelist trusted IP addresses. Taking such an approach and assume that \emph{Alice} would allow herself and \emph{Bob} to query $r_1$, results in the confidentiality policy:
\begin{align*}
    \conf_{white} \eqdef subject() = user(Alice) \lor subject() = user(Bob)
\end{align*}
Note that the equalities in $\conf_{black}$ are concatenated with conjunction whereas they are concatenated with disjunction in $\conf_{white}$.
\end{example}

\begin{example}[Avoid inheritance]\label{ex:avoid}
Let us start off by clarifying that inheritance can not be avoided, however, it is possible to create a confidentiality policy such that it is relevant in the data resource to which it was added but irrelevant when inherited. To achieve this it will be necessary to utilize the $self$ predicate and implication. For a revisit of the $self$ predicate, we point to \autoref{def:pf-user-to-internal}. With this the following confidentiality policy can be created where $\conf_1$ is an arbitrary policy:
\begin{align*}
    \conf \eqdef self \imply \conf_1
\end{align*}
Say $r_1$ was placed in $DR$, shown in \autoref{fig:conf-policy}, with $\conf$. This means that $\conf_1$ should hold under $DR$ with respect to $r_1$ and some context $c$ to successfully query it. However when querying $r_2$ or $r_3$, which has a dependency to $r_1$, $\conf$ is trivially true.
\end{example}

\begin{example}[Author is only reader]\label{ex:conf-reader-author}
A way of ensuring that the author and only the author can query a data resource in a data repository can be achieved by using the \emph{author} attribute name and \emph{subject} function, where the subject can be considered as the reader. Say that $r_1$ is placed in $DR$ as shown in \autoref{fig:conf-policy} with the confidentiality policy:
\begin{align*}
    \conf \eqdef subject() = author
\end{align*}
Then only the author of $r_1$ is allowed to query and read it. The author attribute is added to a data resource automatically during a \emph{put} operation as will be further explained in \autoref{sec:put}. With the given policy $\conf$, $r_1$ can be considered a template and as a result of inheritance, $r_2$ and $r_3$ in $DR$ can only be read by their authors, i.e. the person who performed the \emph{put} to place them in the data repository.
\end{example}

\begin{example}[Anonymized data resources]
Now consider the case of constructing a confidentiality policy that ensures that some property between two data resources holds in $DR$. Say the confidentiality policy $\conf$ is associated to $r_1$, and when querying for $r_3$ it is desired that $\conf$ should ensure that $r_2$ is anonymized. This can be achieved by using the \emph{self} predicate, implication as well as the temporal operators \emph{next} and \emph{until}. Defining the policy as and assuming that $r_2$ has an atomic proposition \emph{anonymized}, it is possible to achieve exactly that:
\begin{align*}
    \conf \eqdef \lnot self \imply \X \left( anonymized \U self \right)
\end{align*}
Let us break it down and justify the construction of it by investigating the behavior when querying $r_1$, $r_2$ and $r_3$. This is under the assumption that a context $c$ is given.

When querying for $r_1$, the left side of the implication is false and the implication is trivially true, thus $DR \models_{c,r_1} \conf$. 

When querying for $r_2$, the left side of the implication is true, so the right side has to be true as well, for the implication to be true. Given that the \emph{next} operator refers to $r_1$, as it is the direct dependency of $r_2$, $DR$ should satisfy $anonymized \U self$ given $r_1$. As \emph{self} is a reference to $r_1$ it follows that $DR \models_{c,r_1} anonymized \U self$ and thus $DR \models_{c,r_2} \conf$.

When querying for $r_3$, the left side of the implication is true, so the right side has to be true as well, for the implication to be true. Given that the \emph{next} operator refers to $r_2$, as it is the direct dependency of $r_3$, $DR$ should satisfy $anonymized \U self$ given $r_2$. The \emph{self} predicate refers to $r_1$ and can not be satisfied in $r_2$, thus the left side of the \emph{until} should be true, which it is as $r_2$ has the atomic proposition $anonymized$. It was shown that $DR \models_{c,r_1} anonymized \U self$ and from this it follows that $DR \models_{c,r_2} anonymized \U self$, which finally means $DR \models_{c,r_3} \conf$.
\end{example}

\subsubsection{Integrity Policies}
Confidentiality policies are concerned with imposing constraints on the information flow in a data repository. On the other hand integrity policies are concerned with guaranteeing provenance insurance whenever data resources are queried in a data repository.
\begin{definition}[Integrity policy]\label{def:ip}
An integrity policy $\inte$ is an UPF $\upf$ where the formulae are constructed from the language described by the grammar in \autoref{tab:pf-grammar-user}.
\end{definition}
It should be noted that like any other LTL formulae or confidentiality policies, multiple policies can be defined and concatenated with conjunction, to construct a single integrity policy.

As already suggested an integrity policy is associated with a \emph{query}, as opposed to a confidentiality policy which is associated to a data resource. This means that an integrity policy is not persistent in the data repository but its lifespan is as long as the \emph{query} operation. This will be explained further in \autoref{sec:query}.

Let us now consider how integrity policies can be constructed by utilizing the temporal and propositional logic operators as well as the newly introduced functions and predicates to the grammar in \autoref{tab:pf-grammar-user}. In the following, a set of examples will be given to illustrate how constraints can express properties of provenance through the domain of integrity policies.

\begin{example}[]\label{ex:mutual-exclusion}
Let us start by considering how one could formulate an integrity policy that ensures that the data resource and its dependencies do not have two given atomic propositions at the same time. Say that it desired to query a data resource if it or any of its dependencies does not contain the atomic proposition $a$ and $b$ at the same time. The following integrity policy $\inte$ specifies exactly that utilizing the temporal operator \emph{always}:
\begin{align*}
    \inte \eqdef  \G \left( \lnot a \lor \lnot b \right)
\end{align*}
Consider the data repository $DR$ shown in \autoref{fig:inte-policy-mutual-exclusion}.
\begin{figure}[!ht]
    \begin{center}
        \begin{forest}
    for tree={circle, draw, s sep=50pt, grow=south, edge={<-}}
    [$r_1$, name=r1, tikz={\node[draw=none, inner sep=0pt, right=2pt of .east]  {\{$a$\}};}
        [$r_2$, name=r2, tikz={\node[draw=none, inner sep=0pt, right=2pt of .east]  {\{$b$\}};}
            [$r_3$, name=r3, tikz={\node[draw=none, inner sep=0pt, right=2pt of .east]  {\{$a$\}};}]
            [$r_4$, name=r4, tikz={\node[draw=none, inner sep=0pt, right=2pt of .east]  {\{$a, b$\}};}]
        ]
    ]
    \draw[->] (r1) to [out=north west,in=north east,looseness=4] (r1);
\end{forest}
        \caption{Data repository containing four data resources $r_1, \ldots, r_4$.}
        \label{fig:inte-policy-mutual-exclusion}
    \end{center}
\end{figure}
Given the data repository $DR$, some context $c$ and the integrity policy $\inte$ it is possible to query for $r_1, r_2$ and $r_3$ as $\inte$ holds under $DR$ with respect to $c$ and those data resources. However $\inte$ does not hold under $DR$ with respect to $r_4$ as $r_4$ contains $a$ and $b$.
\end{example}

\begin{example}[No distrusted author until trusted author]
It could very well be that it is desired to query a data resource with some restrictions on the author of the dependencies. Say that when querying for a data resource $r_i$, one wants the resource to have a dependency to some other resource $r_j$ which was created by the user \emph{Bob}, where no resource in between $r_i$ and $r_j$ was created by the user \emph{Mallory}. Utilizing the \emph{author} attribute name and the \emph{until} operator an integrity policy can be formulated that specifies this:
\begin{align*}
    \inte \eqdef author \neq user(Mallory) \U author = user(Bob)
\end{align*}
Consider the data repository $DR$ given in \autoref{fig:inte-policy-independent-author} and some context $c$.
\begin{figure}[!ht]
    \begin{center}
        \begin{forest}
    for tree={circle, draw, s sep=100pt, grow=south, edge={<-}}
    [$r_1$, name=r1, tikz={\node[draw=none, inner sep=0pt, right=2pt of .east]  {\{$author: Mallory$\}};}
        [$r_2$, name=r2, tikz={\node[draw=none, inner sep=0pt, right=2pt of .east]  {\{$author: Bob$\}};}
            [$r_3$, name=r3, tikz={\node[draw=none, inner sep=0pt, right=2pt of .east]  {\{$author: Charlie$\}};}
                [$r_5$, name=r5, tikz={\node[draw=none, inner sep=0pt, right=2pt of .east]  {\{$author: Alice$\}};}]
            ]
            [$r_4$, name=r4, tikz={\node[draw=none, inner sep=0pt, right=2pt of .east]  {\{$author: Mallory$\}};}
                [$r_6$, name=r6, tikz={\node[draw=none, inner sep=0pt, right=2pt of .east]  {\{$author: Alice$\}};}]
            ]
        ]
    ]
    \draw[->] (r1) to [out=north west,in=north east,looseness=4] (r1);
\end{forest}
        \caption{Data repository containing six data resources $r_1, \ldots, r_6$.}
        \label{fig:inte-policy-independent-author}
    \end{center}
\end{figure}
Let us start by considering if $\inte$ holds under $DR$ when querying for $r_5$. As no data resource on the path from $r_5$ to $r_2$ was created by \emph{Mallory} and $r_2$ was created by \emph{Bob}, then is $\inte$ verified. As $\inte$ is verified when reaching $r_2$ it is unnecessary to consider the remaining of the path being $r_1$. For the same reasons $\inte$ holds under $DR$ when querying for $r_3$ and $r_2$ as well.

Now consider if $\inte$ holds under $DR$ when querying for $r_6$. As the direct dependency $r_4$ of $r_6$ was created by \emph{Mallory} the integrity policy $\inte$ is violated and thus $\inte$ does not hold under $DR$ when querying for $r_6$, as well as for $r_4$.
\end{example}

\begin{example}[Trusted authors]
A data repository is naive in the sense that it does not take the trustworthiness of the authors into consideration, that is up the user performing the query to specify the users that it trust or does not trust. This can be done in a few different ways, all of which has its advantages and disadvantages.

The first way is to take the blacklist approach, the same concept as was used in \autoref{ex:conf-reader-constraints}. Assuming that the user performing the query for a data resource does not trust the users \emph{Mallory} and \emph{Monroe} and thus do not trust any data resources they have placed in the data repository. The following integrity policy $\inte_1$ ensures that a data resource is not returned if it was created by those users or has a dependency to a resource that they created.
\begin{align*}
    \inte_1 \eqdef \G( author \neq user(Mallory) \land author \neq user(Monroe))
\end{align*}

Another way is the whitelist approach which was also introduced in \autoref{ex:conf-reader-constraints}. Assuming that the user performing the query only consider \emph{Alice} and \emph{Bob} for trustworthy authors of data resources and thus only wants to query a data resource if it and all its dependencies were created by one of them. This can be achieved by using the integrity policy $\inte_2$:
\begin{align*}
    \inte_2 \eqdef \G(author = user(Alice) \lor author = user(Bob))
\end{align*}

A third way could be that the user performing the query is not too concerned if the data resource being queried and its dependencies were created by trustworthy authors as long as either the resource itself or one of it dependencies was. Again assuming that \emph{Alice} and \emph{Bob} are considered trustworthy the following integrity policy $\inte_3$ impose exactly that constraint:
\begin{align*}
    \inte_3 \eqdef \F(author = user(Alice) \lor author = user(Bob))
\end{align*}

Finally, it could be that the user does not want data resources created by a distrusted author unless it has been sanitized by a trusted author. Assuming \emph{Mallory} is a distrusted author and \emph{Alice} is a trusted author, the integrity policy $\inte_4$ impose that constraint:
\begin{align*}
    \inte_4 \eqdef \G author \neq user(Mallory) \lor author \neq user(Mallory) \U author = user(Alice)
\end{align*}
\end{example}
\subsection{Operations}
In the following, the domain of operations on a data repository is introduced. The data repository supports two operations, \emph{put} which places a data resource in the repository and thus modifies it, and \emph{query} which retrieves a data resource from the repository but does not remove it and thus \emph{query} does not modify the repository.

\subsubsection{Put}\label{sec:put}
A data repository $\DR$ is initially empty, meaning that $R, \longrightarrow, M, P = \varnothing$. However it can be populated with data resources by utilising the \emph{put} operation. A \emph{put} works directly on the data repository and does not as such need a reference for it.
\begin{definition}[\emph{Put} operation]
A \emph{put} operation is a function $\PUT$ with regards to a data repository $\DR$ where
\begin{itemize}
  \item $r$ is a data resource
  \item $\conf$ is a confidentiality policy
  \item $T : A \rightarrow L$ is a mapping from attribute names to literals
  \item $D \subseteq R$ is a set of dependencies of resource literals
\end{itemize}
\end{definition}
A thing to note is that the dependencies in $D$ are trusted by the system to be correct and it is assumed that they are verified by other means. To consider a \emph{put} successful means that the data resource $r$ is added to the set of data resources $R$ of $DR$. A prerequisite for carrying out a \emph{put} is that $r$ does not already exist in $R$. During the operation, attributes are automatically generated for the data resource $r$. The new attribute names are \emph{author} and \emph{self}. The value of \emph{author} is the subject that is defined in the context, i.e. the author of the data resource is the user who performs the \emph{put}. The value of \emph{self} is the resource itself. This ensures that every data resource in the repository has an author (it was shown in \autoref{ex:conf-reader-author} how it could be used). Furthermore, this ensures that the \emph{self} predicate from the user policy formula grammar in \autoref{tab:pf-grammar-user} can be resolved. Now the inherited confidentiality policies $\iconf$ needs to be considered, meaning that it should be checked if $\iconf(DR,r)$ holds under the data repository given a context of execution and the data resource $r$. Note that the policy is required to be transformed from the UPF grammar to the IPF grammar. If the confidentiality policies hold under the data repository, the data resource is added to the set of resources and the operation can be considered complete.
The semantics of a \emph{put} is formally described in the following definition:
\begin{definition}[Semantics of \emph{put}]
Let $\DR$ be a data repository and $\CON$ be the context of execution of the operation $\PUT$. The confidentiality policy $\conf$ is defined as a UPF and thus needs to be transformed into an IPF using \autoref{def:pf-user-to-internal}. The mapping from attribute names to literals $T$ of \emph{put} is extended with: 
\begin{itemize}
    \item $T = T \cup author \mapsto s$
    \item $T = T \cup self \mapsto r$
\end{itemize}
Then \emph{put} results in a new data repository $DR'=\left(R', \longrightarrow', A', L', M', P' \right)$ where:
\begin{itemize}
    \item $R' = R \cup \{r\}$
    \item $A' = A$
    \item $L' = L$
    \item $M' = M \cup \{ \langle r,a \rangle \mapsto l \mid a \mapsto l \in T \}$
    \item $P' = P \cup \{ r \mapsto [\conf]_{c,r} \}$
\end{itemize}
And the dependency relation $\longrightarrow'$ is:
\begin{itemize}
    \item if $D = \varnothing$, then $\longrightarrow' = \longrightarrow \cup \{r \longrightarrow r\}$ (self-loop)
    \item if $D \neq \varnothing$, then $\longrightarrow' = \longrightarrow \cup \{r \longrightarrow r' \mid r' \in D \land r' \in R \}$
\end{itemize}
The \emph{put} is well-defined if $r$ did not exist in the data repository before the operation, and the inherited confidentiality policy is satisfied:
\begin{itemize}
    \item $r \not\in R$
    \item $DR' \models_{c,r} \iconf(DR',r)$
\end{itemize}
Assume $DR$ is acyclic (apart from self-loops), then $DR'$ is also acyclic, since $r \notin R$ and $D \subseteq R$, as $r$ can only depend on resources already contained in $DR$. Therefore $\iconf(DR,r)$ is well-defined.
\end{definition}

% Explain why confidentiality is checked here
The reason for checking that $DR' \models_{c,r} \iconf(DR',r)$ when $r$ is placed in the data repository, is to avoid that a user can not query the data resource that it just put. Furthermore the check for $DR' \models_{c,r} \iconf(DR',r)$ eliminates the possibility of unsatisfiable confidentiality policies being submitted to the data repository. That is, if a confidentiality policy is not satisfiable under any data repository, any context, and any given data resource, then trivially $DR' \not\models_{c,r} \iconf(DR',r)$. Consider $DR$ as having no data resources with unsatisfiable confidentiality policies, then after \emph{put} neither will $DR'$, since no such resource can be added without violating $DR' \models_{c,r} \iconf(DR',r)$. Let us consider a few examples.

\begin{example}[Unable to query self-placed resource]
Consider that the user \emph{John} had placed a data resource $r$ in a data repository $DR$ with the corresponding confidentiality policy $\conf \eqdef reader() = user(Jane)$. Even though John made the data resource himself, he will never be able to satisfy $\conf$ when querying for $r$, thus making it infeasible for him to read it again.
\end{example}

\begin{example}[Infeasible to query resource]
Consider a data repository $DR$ containing two data resources $r_1$ and $r_2$, where $r_2$ has a dependency to $r_1$ and thus inherit the confidentiality policy $\conf_1$ that is associated to $r_1$. Say $\conf_1 \eqdef \upf$ where $\upf$ is some policy formula and $r_2$ has the confidentiality policy $\conf_2 \eqdef \lnot \upf$ associated to it. When querying for $r_2$ both $\conf_1$ and $\conf_2$ will have to hold in $DR$, i.e. $\iconf(DR,r_2) = \upf \land \lnot \upf$. The two policies clearly contradicts each other and can never be satisfied, thus making it infeasible to query $r_2$ and any data resources that has a dependency to $r_2$.
\end{example}

\subsubsection{Query}\label{sec:query}
Once data has been put in the data repository it might be interesting to query some of the resources again later on. To allow for this the \emph{query} operation has been introduced:
\begin{definition}[\emph{Query} operation]
A \emph{query} operation is a function $\QRY$ with regards to a data repository $DR$ where
\begin{itemize}
  \item $r$ is a data resource
  \item $\inte$ is an integrity policy
\end{itemize}
\end{definition}
As with the \emph{put} operation, an explicit reference for the data repository $DR$ is not needed as it is performed directly on the data repository. For a \emph{query} to be carried out successfully a few things need to be resolved and verified. A \emph{query} is considered successful if the requested data resource is returned from the data repository. A prerequisite for carrying out a \emph{query} is that the data resource exists in the data repository. Now both the integrity and confidentiality policies need to be verified. This means that they have to be transformed from the UPF grammar to corresponding formulae following the IPF grammar. Once that is done it can be checked if the integrity policy and the inherited confidentiality policy of the data resource $\iconf(DR,r)$ hold under the data repository $DR$ given a context of execution $c$ and a data resource $r$. If that is the case the data resource is returned to the user.
The semantics of a \emph{query} is formally described in the following:
\begin{definition}[Semantics of \emph{query}]
Let $\DR$ be a data repository and $\CON$ be the context of execution of the operation $\QRY$. The integrity policy $\inte$ is defined as a UPF and thus needs to be transformed into an IPF using \autoref{def:pf-user-to-internal}.
The \emph{query} is well-defined if $r$ exists in the data repository prior to the operation, and both the inherited confidentiality policy and integrity policy are satisfied:
\begin{itemize}
    \item $r \in R$
    \item $DR \models_{c,r} \iconf(DR, r)$
    \item $DR \models_{c,r} [\inte]_{c,r}$
\end{itemize}
\end{definition}

Regardless of whether a \emph{query} was carried out successfully or not, it does not modify the data repository in any way. Thus \emph{query} will return the queried data resource but it will remain in the data repository. 