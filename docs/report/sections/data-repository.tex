\section{Data Repository}
% How is the data repository modelled.
% Create small (running) example that can be used in the following sections regarding policies

\subsection{Grammar}
% Explain how it extends the grammar of LTL
% Only some propositional logic are directly implemented but the remaining can be used indirectly

\paragraph{self predicate}
\paragraph{user function}
\paragraph{reader function}
\paragraph{literals}
\paragraph{mathematical equality}

\[
\begin{array}{rcll}
    \varphi     & ::=   & a                                 & \text{(atomic proposition)} \\
                & |     & e                                 & \text{(expression)} \\
                & |     & \varphi_1 \vee \varphi_2          & \text{(disjunction)} \\
                & |     & \varphi_1 \wedge \varphi_2        & \text{(conjunction)} \\
                & |     & \varphi_1 \rightarrow \varphi_2   & \text{(implication)} \\
                & |     & \neg \varphi                      & \text{(negation)} \\
                & |     & \bigcirc \varphi                  & \text{(next)} \\
                & |     & \varphi_1 \cup \varphi_2       & \text{(until)} \\
                & |     & \Diamond \varphi                  & \text{(eventually)} \\
                & |     & \square \varphi                   & \text{(always)} \\
    a           & ::=   & self                              & \text{(self predicate)} \\
                & |     & true                              & \text{(true predicate)} \\
                & |     & p                                 & \text{(attribute name)} \\
    e           & ::=   & e_{arg} = e_{arg}                 & \text{(equality)} \\
    %            & |     & u > l                             & \text{(greater than)} \\
    %            & |     & u \geq l                          & \text{(greater than or equal)} \\
    %            & |     & u < l                             & \text{(less than)} \\
    %            & |     & u \leq l                          & \text{(less than or equal)} \\
    e_{arg}     & ::=   & a                                 & \text{(atomic proposition)} \\
                & |     & f                                 & \text{(function)} \\
                & |     & l                                 & \text{(literal)} \\
    f           & ::=   & user(s)                           & \text{(user function)} \\
                & |     & reader()                          & \text{(reader function)} \\
    l           & ::=   & s                                 & \text{(string literal)} \\
                & |     & n                                 & \text{(number literal)} \\
                & |     & b                                 & \text{(boolean literal)}
\end{array}
\]

\paragraph{Limitations of grammar}
% Write about this if there something relevant

\subsection{Confidentiality Policies}
% What is meant by a confidentiality policy
% Compare with Role-Based Access Control (RBAC)
% - RBAC is a simple way to provide confidentiality
% Compare with Access-Control List (ACL)
% White and black lists of who can use your data can be achieved by author = "username" and author = !"username" respectively
% Continue running example

\subsection{Integrity Policies}
% What is meant by an integrity policy
% Continue running example