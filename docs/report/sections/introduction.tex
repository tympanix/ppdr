\section{Introduction}
% What is the problem?
% Why is it useful
% Get the reader excited
% - Could consider to introduce the real world systems
% Make a citation for the github repo: https://zenodo.org/
In this day and age, news regarding data leaks and data misuses are unfortunately becoming more common than ever. An example of the latter back from 2012, was the case where the United States major retailer Target Corporation used customers' shopping history to identify pregnant customers, as they are considered high-value customers. This caused a backlash for the retailer as an unknowingly pregnant high school student started to receive personalised coupons for baby equipment\footnote{\url{https://www.forbes.com/sites/kashmirhill/2012/02/16/how-target-figured-out-a-teen-girl-was-pregnant-before-her-father-did/}}. In an attempt to overcome the data misuses, amongst other things, the European Union composed the infamous General Data Protection Regulation (Regulation (EU) 2016/679) (GDPR). It has been less than two years since GDPR was implemented in the European countries around May, 2018\footnote{\url{https://www.dlapiperdataprotection.com/index.html?t=law&c=DK}}. Since then has research groups formed. CyberSec4Europe is one of them, where their goal is ``designing, testing and demonstrating potential governance structures for a future European Cybersecurity Competence Network''\footnote{\url{https://cybersec4europe.eu/about/}}. Inspired by CyberSec4Europe's and GDPR's goal, this paper investigates the possibility to create a data repository abstraction, where the data resources in the repository have dependencies to each other, corresponding to their origins. Furthermore in order to create restrictions about who can access the data resources as well as creating restrictions about the quality of the data resources, we propose a grammar that extends the grammar of Linear Temporal Logic (LTL). The new grammar is going to describe what we call Policy Formulae (PF) and makes it possible, amongst other things, to make properties about the subject performing a put or query operation on the data repository. An advantage of extending the LTL grammar allows the PF to impose the policies not only on data resources themselves, but on their dependencies.

\paragraph{Contributions:} The three main contributions of the paper are as follows: (i) we extend the grammar for LTL to create a new grammar for PF that allows one to set up policies for data resources in a data repository, as well as a method for representing a PF as an LTL formula. (ii) We present an abstraction for a data repository that can be transformed into a transition system. (iii) We offer an implementation of the data repository in Google's programming language Go\footnote{\url{https://golang.org/}} where, amongst other features, any PF can be checked for satisfiability. The paper also provides one minor contribution: (i) We present an algorithm for determining elementary sets, that in theory is no better than the naive approach but in practice shows to be a significant improvement.

\paragraph{Paper Structure:}
The structure of the paper is as follows. First are some preliminaries covered in \autoref{sec:preliminaries}, which introduces some theory, terminology and properties regarding LTL formulae. Next is the data repository introduced in \autoref{sec:data-repository}, where the structure of the data repository and the operations to perform on the repository is shown. Furthermore is the grammar to construct policy formulae covered as well as the two types of policies that are created by policy formulae. In \autoref{sec:methods} are the parsing of policy formulae described as well as the method used for checking whether a formulae is satisfied or not. This is followed by a discussion in \autoref{sec:discussion} where justifications for using one approach over another and more are covered. Finally are ideas for future work to the data repository described in \autoref{sec:future-work}