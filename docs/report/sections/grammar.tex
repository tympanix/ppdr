% Content:
% - Explain how it extends the grammar of LTL
% - Only some propositional logic are directly implemented but the remaining can be used indirectly
% - Explain that the grammar is abstract and somethings are not explicitly mentioned
\subsection{Grammar}\label{sec:grammar}
In this section a new type of formula is introduced named Policy Formula ($PF$) which is inspired by LTL, which was introduced in \autoref{sec:ltl}. This type of formula is used to formulate confidentiality and integrity policies in a data repository. We distinct between two types of representations of Policy Formulae, each defined by their own grammar, called User Policy Formulae ($UPF$) and Internal Policy Formulae ($IPF$). This abstraction is necessary to distinguish between the formulae as presented externally to the actor of the system and the internal representation which is exposed to the model checker. The grammar for $UPF$ can be seen in \autoref{tab:pf-grammar-user} and for $IPF$ in \autoref{tab:pf-grammar}.

\begin{table}[!ht]
    \centering
    $
    \begin{array}{rcll}
        \pf_u   & ::=   & true              & \text{(true predicate)} \\
                & |     & \pfn              & \text{(extension)} \\
                & |     & \pf_1 \land \pf_2 & \text{(conjunction)} \\
                & |     & \pf_1 \lor \pf_2  & \text{(disjunction)} \\
                & |     & \pf_1 \imply \pf_2 & \text{(implication)} \\
                & |     & \neg \pf          & \text{(negation)} \\
                & |     & \X \pf            & \text{(next)} \\
                & |     & \pf_1 \U \pf_2    & \text{(until)} \\
                & |     & \F \pf            & \text{(eventually)} \\
                & |     & \G \pf            & \text{(always)} \\
        \pfn    & ::=   & e_1 \bowtie e_2   & \text{(expression)} \\
                & |     & ap                & \text{(atomic proposition)} \\
                & |     & self              & \text{(self predicate)} \\
        f       & ::=   & user(str)           & \text{(user function)} \\
                & |     & subject()         & \text{(subject function)} \\
        e       & ::=   & atr               & \text{(attribute name)} \\
                & |     & f                 & \text{(functions)} \\
                & |     & l                 & \text{(literal)} \\
        \bowtie & ::=   & =                 & \text{(equal)} \\
                & |     & \neq              & \text{(not equal)} \\
                & |     & <                 & \text{(less than)} \\
                & |     & \leq              & \text{(less than equal)} \\
                & |     & >                 & \text{(greater than)} \\
                & |     & \geq              & \text{(greater than equal)} \\
        l       & ::=   & str               & \text{(string literal)} \\
                & |     & num               & \text{(number literal)} \\
                & |     & bol               & \text{(boolean literal)} \\
    \end{array}
    $
    \caption{Grammar for user policy formula}
    \label{tab:pf-grammar-user}
\end{table}

\begin{table}[!ht]
    \centering
    $
    \begin{array}{rcll}
        \pf     & ::=   & true              & \text{(true predicate)} \\
                & |     & \pfn              & \text{(extension)} \\
                & |     & \pf_1 \land \pf_2 & \text{(conjunction)} \\
                & |     & \neg \pf          & \text{(negation)} \\
                & |     & \X \pf            & \text{(next)} \\
                & |     & \pf_1 \U \pf_2    & \text{(until)} \\
        \pfn    & ::=   & e_1 \bowtie e_2   & \text{(expression)} \\
        e       & ::=   & atr               & \text{(attribute name)} \\
                & |     & l                 & \text{(literal)} \\
        \bowtie & ::=   & =                 & \text{(equal)} \\
                & |     & \neq              & \text{(not equal)} \\
                & |     & <                 & \text{(less than)} \\
                & |     & \leq              & \text{(less than equal)} \\
                & |     & >                 & \text{(greater than)} \\
                & |     & \geq              & \text{(greater than equal)} \\
        l       & ::=   & str               & \text{(string literal)} \\
                & |     & num               & \text{(number literal)} \\
                & |     & bol               & \text{(boolean literal)} \\
                & |     & usr               & \text{(user literal)} \\
                & |     & rsc               & \text{(resource literal)} \\
                & |     & nil               & \text{(nil literal)}
    \end{array}
    $
    \caption{Grammar for Internal Policy Formula.}
    \label{tab:pf-grammar}
\end{table}

To couple the two types of Policy Formulae we define a translation from $UPF$ to $IPF$. This transformation replaces syntax from the grammar of $UPF$ with equivalent grammar of $IPF$ to produce a more concise representation. These replacements include \emph{syntactic sugar}, i.e. syntax which is exposed to the actor of the system but can be expressed with existing syntax without loss of semantics. Specifically for $IPF$ we introduce user literals ($usr$) and resource literals ($rcs$), which are not allowed in $UPF$, i.e. manipulation and construction of these literals are only allowed in the internal representation. It should be noted that the resource literal ($rcs$) can be expressed equivalently with a data resource $r \in R$ of a data repository $\DR$. In $UPF$ one resolves to the $subject()$ function, which resolves the user literal for the current subject performing an action on the data repository.

\begin{definition}[Transformation from UPF to IPF]\label{def:pf-user-to-internal}
Given a data resource $r$ and an execution context $\CON$ the UPD $\upf$ can be transformed to an IPF $\pf$, denoted $\pf = [\pf_u]_{c,r}$ defined by the following:
\begin{itemize}
    \item $[true]_{c,r} \eqdef true$
    \item $[\pf_{u_1} \land \pf_{u_2}]_{c,r} \eqdef [\pf_{u_1}]_{c,r} \land [\pf_{u_2}]_{c,r}$
    \item $[\pf_{u_1} \lor \pf_{u_2}]_{c,r} \eqdef \lnot(\lnot [\pf_{u_1}]_{c,r} \land \lnot [\pf_{u_1}]_{c,r})$
    \item $[\pf_{u_1} \imply \pf_{u_2}]_{c,r} \eqdef \lnot([\pf_{u_1}]_{c,r} \land \lnot [\pf_{u_1}]_{c,r})$
    \item $[\lnot \pf_{u}]_{c,r} \eqdef \lnot [\pf_{u}]_{c,r}$
    \item $[\X \pf_{u}]_{c,r} \eqdef \X [\pf_{u}]_{c,r}$
    \item $[\pf_{u_1} \U \pf_{u_2}]_{c,r} \eqdef [\pf_{u_1}]_{c,r} \U [\pf_{u_2}]_{c,r}$
    \item $[\F \pf_{u}]_{c,r} \eqdef true \U [\pf_{u}]_{c,r}$ \hfill(using \autoref{eq:eventually})
    \item $[\G \pf_{u}]_{c,r} \eqdef \lnot (true \U \lnot [\pf_{u}]_{c,r})$ \hfill(using \autoref{eq:eventually} and \autoref{eq:always})
    \item $[e_1 \bowtie e_2]_{c,r} \eqdef [e_1]_{c,r} \bowtie [e_2]_{c,r}$
    \item $[ap]_{c,r} \eqdef atr = true$ where $atr \eqdef ap$
    \item $[self]_{c,r} \eqdef self = rsc$ where $rsc = r$
    \item $[user(str)]_{c,r} \eqdef
        \begin{cases*}
            N(str)  & iff $str \in I$ \\
            nil     & otherwise
        \end{cases*}$
    \item $[subject()]_{c,r} \eqdef usr$ where $usr = s$
    \item $[atr]_{c,r} \eqdef atr$
    \item $[l]_{c,r} \eqdef l$
\end{itemize}
\end{definition}

As can be seen from \autoref{def:pf-user-to-internal} the transformation is nothing more than a recursive replacement of syntax using the context to resolve operators which are sensitive to the context. Notice that, contrary to LTL, $IPF$ represents atomic propositions as an equality with the boolean literal $true$. Furthermore the $self$ predicate, which has a special meaning in $UPF$, can in the same manner be represented as an equality in $IPF$. Since every predicate of an $IPF$ is of the form $\pfn$, i.e. expressed using some equality of the form $e_1 \bowtie e_2$, the satisfaction relation is defined below for a data resource $r \in R$ belonging to $\DR$:
\begin{align*}
    r &\models e_1 \bowtie e_2 &\text{iff }& eval(e_1, r) [\![ \bowtie ]\!] eval(e_2, r)
\end{align*}
and the evaluation function, $eval : e \times r \rightarrow l$, which is defined by the following:
\begin{align*}
    eval(e,r) =
    \begin{cases*}
        l   & if $e$ is type $atr$ and $\langle r,atr\rangle \mapsto l \in M$ \\
        nil & if $e$ is type $atr$ and $\langle r,atr\rangle \mapsto l \not\in M$ \\
        e         & otherwise (literal)
    \end{cases*}
\end{align*}
By the evaluation of the symbol $e$ we simply distinguish between attribute names, which has be resolved by the mapping $M$ for a given data resource, and literals, which represent themselves. Attributes which are not found in $M$ is evaluated as the $nil$ literal. The attribute names are what allows the actor of the system to reason about any single data resource. Since the mapping $M$ maps to literals, the $\bowtie$ operator is limited to this domain. The semantics for $\bowtie$ is defined following the definitions below. Some types, i.e. numbers, booleans and strings, are defined following the partial ordering of their respective elements.
\begin{align*}
    e_1 [\![ \bowtie ]\!] e_2 \imply& \enskip false \text{ iff } type(e_1) \neq type(e_2) \\
    str [\![ \bowtie ]\!] str' \imply& \enskip \text{lexicographical ordering of strings} \\
    num [\![ \bowtie ]\!] num' \imply& \enskip \text{ordering of numbers in } \mathbb{R} \\
    bol [\![ \bowtie ]\!] bol' \imply& \enskip \text{ordering of booleans in } \mathbb{B} \\
    usr [\![ \bowtie ]\!] usr' \imply&
        \begin{cases*}
            usr = usr'      & if $\bowtie$ is type $=$ (equal) \\
            usr \neq usr'   & if $\bowtie$ is type $\neq$ (not equal) \\
            false       & otherwise
        \end{cases*} \\
    rsc [\![ \bowtie ]\!] rsc' \imply&
        \begin{cases*}
            rsc = rsc'      & if $\bowtie$ is type $=$ (equal) \\
            rsc \neq rsc'   & if $\bowtie$ is type $\neq$ (not equal) \\
            false       & otherwise
        \end{cases*} \\
    nil [\![ \bowtie ]\!] nil' \imply& false
\end{align*}

This concludes the semantic definition of $UPF$ and $IPF$ and how they are evaluated. Lastly we introduce a final transformation from $IPF$ to $LTL$, which allows an $IPF$ formula to be translated and provided to well-known $LTL$ model checking algorithms (these methods algorithms are explored in \autoref{sec:methods}). The approach is to substitute every occurrence of $e_1 \bowtie e_1$ with a new atomic proposition $ap_{e_1 \bowtie e_2}$, which represents the evaluation of the equality. This means that $ap_{e_1 \bowtie e_2}$ is only satisfied for $r \in R$ given $\DR$ iff $r \models e_1 \bowtie e_2$. The transformation from $IPF$ $\pf$ to $LTL$ formula $\phi$ is provided using the following definition:
\begin{definition}[Transformation from $IPF$ to $LTL$]\label{def:pf-to-ltl}
%Given a data repository $\DR$, a data resource $r \in R$ and an $IPF$ $\pf$, the transformation into a $LTL$ formula $\phi$ over the set of atomic propositions $AP_\phi$ and transformation table $\tau : AP_\phi \rightarrow e_1 \bowtie e_2$ is performed by the following. For each $e_1 \bowtie e_2$ in $\pf$ introduce the new atomic proposition and map the atomic proposition in $\tau$:
Given an $IPF$ $\pf$, the transformation into a $LTL$ formula $\phi$ over the set of atomic propositions $AP_\phi$ and transformation table $\tau : AP_\phi \rightarrow e_1 \bowtie e_2$ is performed by the following. For each $e_1 \bowtie e_2$ in $\pf$ introduce the new atomic proposition and map the atomic proposition in $\tau$:
\begin{align*}
    AP_\phi = AP_\phi \cup \left\{ap_{e_1 \bowtie e_2}\right\} \enskip \text{and} \enskip 
    \tau \cup \{ap_{e_1 \bowtie e_2} \mapsto e_1 \bowtie e_2\}
\end{align*}
Then perform the following substitution:
\begin{align*}
    \pf[^{ap_{e_1 \bowtie e_2}}/_{e_1 \bowtie e_2}]
\end{align*}
\end{definition}
The purpose of $\tau$, as described in \autoref{def:pf-to-ltl}, is to provide a mapping from the newly introduced atomic propositions $ap_{e_1 \bowtie e_2}$ to the substituted subformula $e_1 \bowtie e_2$. One can then satisfy $ap_{e_1 \bowtie e_2}$ iff $r \models \tau(ap_{e_1 \bowtie e_2})$.