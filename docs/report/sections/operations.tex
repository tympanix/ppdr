\subsection{Operations}
In the following, the domain of operations on a data repository is introduced. The data repository supports two operations, \emph{put} which places a data resource in the repository and thus modifies it, and \emph{query} which retrieves a data resource from the repository but does not remove it and thus \emph{query} does not modify the repository.

\subsubsection{Put}\label{sec:put}
A data repository $\DR$ is initially empty, meaning that $R, \longrightarrow, M, P = \varnothing$. However it can be populated with data resources by utilizing the \emph{put} operation. A \emph{put} works directly on the data repository and does not as such need a reference for it.
\begin{definition}[\emph{Put} operation]
A \emph{put} operation is a function $\PUT$ with regards to a data repository $\DR$ where
\begin{itemize}
  \item $r$ is a data resource
  \item $\conf$ is a confidentiality policy
  \item $T : A \rightarrow L$ is a mapping from attribute names to literals
  \item $D \subseteq R$ is a set of dependencies of resource literals
\end{itemize}
\end{definition}
A thing to note is that the dependencies in $D$ are trusted by the system to be correct and it is assumed that they are verified by other means. To consider a \emph{put} successful means that the data resource $r$ is added to the set of data resources $R$ of $DR$. A prerequisite for carrying out a \emph{put} is that $r$ does not already exist in $R$. During the operation, attributes are automatically generated for the data resource $r$. The new attribute names are \emph{author} and \emph{self}. The value of \emph{author} is the subject that is defined in the context, i.e. the author of the data resource is the user who performs the \emph{put}. The value of \emph{self} is the resource itself. This ensures that every data resource in the repository has an author (it was shown in \autoref{ex:conf-reader-author} how it could be used). Furthermore, this ensures that the \emph{self} predicate from the user policy formula grammar in \autoref{tab:pf-grammar-user} can be resolved. Now the inherited confidentiality policies $\iconf$ needs to be considered, meaning that it should be checked if $\iconf(DR,r)$ holds under the data repository given a context of execution and the data resource $r$. Note that the policy is required to be transformed from the UPF grammar to the IPF grammar. If the confidentiality policies hold under the data repository, the data resource is added to the set of resources and the operation can be considered complete.
The semantics of a \emph{put} is formally described in the following definition:
\begin{definition}[Semantics of \emph{put}]
Let $\DR$ be a data repository and $\CON$ be the context of execution of the operation $\PUT$. The confidentiality policy $\conf$ is defined as a UPF and thus needs to be transformed into an IPF using \autoref{def:pf-user-to-internal}. The mapping from attribute names to literals $T$ of \emph{put} is extended with: 
\begin{itemize}
    \item $T = T \cup author \mapsto s$
    \item $T = T \cup self \mapsto r$
\end{itemize}
Then \emph{put} results in a new data repository $DR'=\left(R', \longrightarrow', A', L', M', P' \right)$ where:
\begin{itemize}
    \item $R' = R \cup \{r\}$
    \item $A' = A$
    \item $L' = L$
    \item $M' = M \cup \{ \langle r,a \rangle \mapsto l \mid a \mapsto l \in T \}$
    \item $P' = P \cup \{ r \mapsto [\conf]_{c,r} \}$
\end{itemize}
And the dependency relation $\longrightarrow'$ is:
\begin{itemize}
    \item if $D = \varnothing$, then $\longrightarrow' = \longrightarrow \cup \{r \longrightarrow r\}$ (self-loop)
    \item if $D \neq \varnothing$, then $\longrightarrow' = \longrightarrow \cup \{r \longrightarrow r' \mid r' \in D \land r' \in R \}$
\end{itemize}
The \emph{put} is well-defined if $r$ did not exist in the data repository before the operation, and the inherited confidentiality policy is satisfied:
\begin{itemize}
    \item $r \not\in R$
    \item $DR' \models_{c,r} \iconf(DR',r)$
\end{itemize}
Assume $DR$ is acyclic (apart from self-loops), then $DR'$ is also acyclic, since $r \notin R$ and $D \subseteq R$, as $r$ can only depend on resources already contained in $DR$. Therefore $\iconf(DR,r)$ is well-defined.
\end{definition}

% Explain why confidentiality is checked here
The reason for checking that $DR' \models_{c,r} \iconf(DR',r)$ when $r$ is placed in the data repository, is to avoid that a user can not query the data resource that it just put. Furthermore the check for $DR' \models_{c,r} \iconf(DR',r)$ eliminates the possibility of unsatisfiable confidentiality policies being submitted to the data repository. That is, if a confidentiality policy is not satisfiable under any data repository, any context, and any given data resource, then trivially $DR' \not\models_{c,r} \iconf(DR',r)$. Consider $DR$ as having no data resources with unsatisfiable confidentiality policies, then after \emph{put} neither will $DR'$, since no such resource can be added without violating $DR' \models_{c,r} \iconf(DR',r)$. Let us consider a few examples.

\begin{example}[Unable to query self-placed resource]
Consider that the user \emph{John} had placed a data resource $r$ in a data repository $DR$ with the corresponding confidentiality policy $\conf \eqdef reader() = user(Jane)$. Even though John made the data resource himself, he will never be able to satisfy $\conf$ when querying for $r$, thus making it infeasible for him to read it again.
\end{example}

\begin{example}[Infeasible to query resource]
Consider a data repository $DR$ containing two data resources $r_1$ and $r_2$, where $r_2$ has a dependency to $r_1$ and thus inherit the confidentiality policy $\conf_1$ that is associated to $r_1$. Say $\conf_1 \eqdef \upf$ where $\upf$ is some policy formula and $r_2$ has the confidentiality policy $\conf_2 \eqdef \lnot \upf$ associated to it. When querying for $r_2$ both $\conf_1$ and $\conf_2$ will have to hold in $DR$, i.e. $\iconf(DR,r_2) = \upf \land \lnot \upf$. The two policies clearly contradicts each other and can never be satisfied, thus making it infeasible to query $r_2$ and any data resources that has a dependency to $r_2$.
\end{example}

\subsubsection{Query}\label{sec:query}
Once data has been put in the data repository it might be interesting to query some of the resources again later on. To allow for this the \emph{query} operation has been introduced:
\begin{definition}[\emph{Query} operation]
A \emph{query} operation is a function $\QRY$ with regards to a data repository $DR$ where
\begin{itemize}
  \item $r$ is a data resource
  \item $\inte$ is an integrity policy
\end{itemize}
\end{definition}
As with the \emph{put} operation, an explicit reference for the data repository $DR$ is not needed as it is performed directly on the data repository. For a \emph{query} to be carried out successfully a few things need to be resolved and verified. A \emph{query} is considered successful if the requested data resource is returned from the data repository. A prerequisite for carrying out a \emph{query} is that the data resource exists in the data repository. Now both the integrity and confidentiality policies need to be verified. This means that they have to be transformed from the UPF grammar to corresponding formulae following the IPF grammar. Once that is done it can be checked if the integrity policy and the inherited confidentiality policy of the data resource $\iconf(DR,r)$ hold under the data repository $DR$ given a context of execution $c$ and a data resource $r$. If that is the case the data resource is returned to the user.
The semantics of a \emph{query} is formally described in the following:
\begin{definition}[Semantics of \emph{query}]
Let $\DR$ be a data repository and $\CON$ be the context of execution of the operation $\QRY$. The integrity policy $\inte$ is defined as a UPF and thus needs to be transformed into an IPF using \autoref{def:pf-user-to-internal}.
The \emph{query} is well-defined if $r$ exists in the data repository prior to the operation, and both the inherited confidentiality policy and integrity policy are satisfied:
\begin{itemize}
    \item $r \in R$
    \item $DR \models_{c,r} \iconf(DR, r)$
    \item $DR \models_{c,r} [\inte]_{c,r}$
\end{itemize}
\end{definition}

Regardless of whether a \emph{query} was carried out successfully or not, it does not modify the data repository in any way. Thus \emph{query} will return the queried data resource but it will remain in the data repository. 